\documentclass{article}

\usepackage[italian]{babel}
\usepackage[T1]{fontenc}
\usepackage[utf8]{inputenc}
\usepackage{graphicx}
\usepackage{wrapfig}

\setlength{\parindent}{0pt}

\usepackage{geometry}
\geometry{
	paper=a4paper, % Paper size, change to letterpaper for US letter size
	top=3.25cm, % Top margin
	bottom=4cm, % Bottom margin
	left=3.5cm, % Left margin
	right=3.5cm, % Right margin
	headheight=0.75cm, % Header height
	footskip=1cm, % Space from the bottom margin to the baseline of the footer
	headsep=0.75cm, % Space from the top margin to the baseline of the header
	%showframe, % Uncomment to show how the type block is set on the page
}

\usepackage[semibold]{ebgaramond}
\usepackage{sectsty} % Allows changing the font options for sections in a document

\sectionfont{\fontsize{13.5pt}{18pt}\selectfont} % Set font options for sections
%\subsectionfont{\mdseries\scshape\normalsize} % Set font options for subsections
\subsubsectionfont{\mdseries\upshape\bfseries\normalsize} % Set font options for subsubsections

\usepackage{marginnote} % Required to output text in the margin

\newcommand{\years}[1]{\marginnote{\small #1}} % New command for adding years to the margin
\renewcommand*{\raggedleftmarginnote}{} % Left-align the years in the margin
\setlength{\marginparsep}{-10pt} % Move the margin content closer to the text
\reversemarginpar % Margin text to be output into the left margin instead of the default right margin

\usepackage[usenames, dvipsnames]{xcolor} % Required for specifying colours by name
\usepackage[bookmarks, colorlinks, breaklinks]{hyperref} % Required for links

% Set link colours
\hypersetup{
	linkcolor=blue,
	citecolor=blue,
	filecolor=black,
	urlcolor=MidnightBlue
}

\hypersetup{
	pdftitle={Filippo Barbari - Curriculum vitae},
	pdfauthor={Filippo Barbari}
}

\newcommand{\referenza}[6]{
	\textbf{#1}
	\begin{itemize}
		\setlength\itemsep{0em}
		\item #2
		\item[-] Mail: #3
		\item[*] Telefono: #4
		\item[.] Sito: #5
		\item[] \textit{#6}
	\end{itemize}
}

\newcommand{\https}[1]{\href{https://#1}{#1}}

\newcommand{\exam}[2]{\textbf{#1}  #2/30}

\newcommand{\skill}[2]{\texttt{#1} #2/10}

\begin{document}
	
	\begin{wrapfigure}[0]{r}{0.2\textwidth}
		\includegraphics[width=0.18\textwidth]{fototessera}
	\end{wrapfigure}
		
	{\LARGE\bfseries Filippo Barbari} % Name
	\bigskip
	
	Via Emilia 42\\ % Address
	47838 Riccione (RN), Italia
	\medskip % Whitespace
	
	Telefono: 327-2207037\\
	Fisso: 0541-644233
	\medskip
	
	\noindent
	\begin{tabular}{ll}
		Email: & \href{mailto:filippo.barbari@gmail.com}{filippo.barbari@gmail.com}\\
		Email istituzionale: & \href{mailto:filippo.barbari@studio.unibo.it}{filippo.barbari@studio.unibo.it}
	\end{tabular}
	\medskip
	
	Nato: 24 agosto 1999, Rimini (RN), Italia\\ % Date of birth
	Nazionalità: Italiana % Nationality
	
	\section*{Referenze}
	
	\referenza{Prof. Moreno Marzolla}
	{Professore associato del Dipartimento di Informatica - Scienza e Ingegneria (DISI), Università di Bologna, Italia.}
	{\href{mailto:moreno.marzolla@unibo.it}{moreno.marzolla@unibo.it}}
	{+39 0547 338861}
	{\href{https://www.moreno.marzolla.name}{www.moreno.marzolla.name}}
	{\textit{Il Prof. Marzolla è stato il relatore della mia tesi di laurea Triennale e il tutor del mio tirocinio curriculare.}}
	
	\section*{Formazione}
	\years{2013-2018} \textbf{Diploma di Maturità} conseguito con votazione 90/100 presso Liceo Scientifico A. Volta, Riccione (RN)\\
	
	\years{2018-2021} \textbf{Laurea Triennale} in Ingegneria e Scienze Informatiche conseguita con votazione 106/110 presso Alma Mater Studiorum Università di Bologna, Campus di Cesena in data 7 ottobre 2021\\
	
	\years{2021-oggi} \textbf{Laurea Magistrale} in Ingegneria e Scienze Informatiche presso Alma Mater Studiorum Università di Bologna, Campus di Cesena
	
	\subsection*{Esami sostenuti durante il corso di laurea triennale}
	\begin{tabular}{ll}
		\textbf{Programmazione} & 28/30\\
		\textbf{Architetture degli Elaboratori} & 29/30\\
		\textbf{Analisi Matematica} & 27/30\\
		\textbf{Algoritmi e Strutture Dati} & 30/30 con lode\\
		\textbf{Algebra e Geometria} & 26/30\\
		\textbf{Programmazione di Reti} & 28/30\\
		\textbf{Programmazione ad Oggetti} & 28/30\\
		\textbf{Metodi Numerici} & 22/30\\
		\textbf{Matematica Discreta e Probabilità} & 25/30\\
		\textbf{Fisica} & 30/30 con lode\\
		\textbf{Basi di Dati} & 30/30\\
		\textbf{Sistemi Operativi} & 24/30\\
		\textbf{Computer Graphics} & 28/30\\
		\textbf{Crittografia} & 30/30\\
		\textbf{High-Performance Computing} & 30/30\\
		\textbf{Ingegneria del software} & 23/30\\
		\textbf{Programmazione di Applicazioni Data-Intensive} & 27/30\\
		\textbf{Reti di telecomunicazione} & 23/30\\
		\textbf{Ricerca operativa} & 26/30\\
		\textbf{Tecnologie web} & 24/30\\
	\end{tabular}

	\subsection*{Tirocinio curriculare}
	Sviluppo e valutazione delle prestazioni di un'applicazione parallela per simulazioni tecnico-scientifiche.
	
	Tutor: Prof. Moreno Marzolla
	
	Lo scopo di questo tirocinio è stato lo sviluppo di una simulazione per il calcolo dell'impacchettamento ottimale di sfere rappresentanti particelle di carburante all'interno di un serbatoio cubico.
	
	\subsection*{Tesi di Laurea Triennale}
	Implementazione CUDA dell'algoritmo di Bellman-Ford.
	
	Relatore: Prof. Moreno Marzolla
	
	Progettazione, sviluppo e valutazione delle prestazioni di tre differenti implementazioni parallele dell'algoritmo di Bellman-Ford su GPU.
	
	Votazione finale: 106/110
	
	Link: \https{amslaurea.unibo.it/24313}
	
	Codice: \https{github.com/Ledmington/bellman-ford-cuda}
	
	\subsection*{Esami sostenuti durante il corso di laurea magistrale}
	\noindent{\small Ultimo aggiornamento: 27/01/2022}
	\\
	\begin{tabular}{ll}
		\textbf{Machine Learning} & 29/30\\
	\end{tabular}
	
	\section*{Lingue}
	Italiano: madrelingua\\
	Inglese: FCE livello B-2
	
	\section*{Skill}
	\skill{C}{9}\\
	\skill{C++}{6}\\
	\skill{Java}{7}
	
	\vfill
	\begin{center}
		\scriptsize
		Ultimo aggiornamento: \today
	\end{center}
	
\end{document}