\documentclass[curriculum-vitae-ita]{subfiles}

\begin{document}
	\section*{Formazione}
		\years{2021 - oggi} \textbf{Laurea Magistrale} in Ingegneria e Scienze Informatiche presso Alma Mater Studiorum Università di Bologna, Campus di Cesena, Italia.\\
		
		\years{2018 - 2021} \textbf{Laurea Triennale} in Ingegneria e Scienze Informatiche conseguita con votazione 106/110 in data 7 ottobre 2021 presso Alma Mater Studiorum Università di Bologna, Campus di Cesena, Italia.\\

		\years{2013 - 2018} \textbf{Diploma di Maturità} conseguito con votazione 90/100 presso Liceo Scientifico A. Volta, Riccione (RN), Italia.
		
		\iffalse
		\subsection*{Esami sostenuti durante il corso di laurea triennale}
			\begin{minipage}[t]{.47\textwidth}
				\exam{Algebra e Geometria}{26}
				\exam{Algoritmi e Strutture Dati}{31}
				\exam{Analisi Matematica}{27}
				\exam{Architetture degli Elaboratori}{29}
				\exam{Basi di Dati}{30}
				\exam{Computer Graphics}{28}
				\exam{Crittografia}{30}
				\exam{Fisica}{31}
				\exam{High-Performance Computing}{30}
				\exam{Ingegneria del software}{23}
				\exam{Matematica Discreta e Probabilità}{25}
			\end{minipage}
			\hfill
			\begin{minipage}[t]{.47\textwidth}
				\exam{Metodi Numerici}{22}
				\exam{Programmazione}{28}
				\exam{Programmazione ad Oggetti}{28}
				\exam{Programmazione di Applicazioni\\Data-Intensive}{27}
				\exam{Programmazione di Reti}{28}
				\exam{Reti di telecomunicazione}{23}
				\exam{Ricerca operativa}{26}
				\exam{Sistemi Operativi}{24}
				\exam{Tecnologie web}{24}
			\end{minipage}
		\fi
		
		\subsection*{Tesi di Laurea Triennale}
			\begin{itemize}
				\item[-] {\large Titolo:} Implementazione CUDA dell'algoritmo di Bellman-Ford.
				\item[$\star$] {\large Relatore:} Prof. Moreno Marzolla
				\item {\large Presso:} Dipartimento di Informatica - Scienze e Ingegneria (DISI), Università di Bologna, Italia
				\item[$\circ$] {\large Votazione:} 106/110
				\item[] Progettazione, sviluppo e valutazione delle prestazioni di tre differenti implementazioni parallele dell'algoritmo di Bellman-Ford su GPU.
				\item[] Tesi disponibile \href{https://amslaurea.unibo.it/24313}{qui}.
				\item[] Codice disponibile \href{https://github.com/Ledmington/bellman-ford-cuda}{qui}.
			\end{itemize}
		
		\subsection*{Tirocinio curriculare}
			\begin{itemize}
				\item[-] {\large Titolo:} Sviluppo e valutazione delle prestazioni di un'applicazione parallela per simulazioni tecnico-scientifiche.
				\item[$\star$] {\large Tutor:} Prof. Moreno Marzolla
				\item {\large Presso:} Dipartimento di Informatica - Scienze e Ingegneria (DISI), Università di Bologna, Italia
				\item[] Lo scopo di questo tirocinio è stato lo sviluppo di una simulazione per il calcolo dell'impacchettamento ottimale di sfere rappresentanti particelle di carburante all'interno del serbatoio di un razzo.
			\end{itemize}
		
		\iffalse
		\subsection*{Esami sostenuti durante il corso di laurea magistrale}
			{\small Ultimo aggiornamento: 27/01/2022}
			\medskip
			
			\begin{minipage}[t]{.47\textwidth}
				\exam{Machine Learning}{29}
			\end{minipage}
			\hfill
			\begin{minipage}[t]{.47\textwidth}
				% aggiungere qui futuri esami
			\end{minipage}
		\fi
\end{document}